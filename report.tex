\documentclass[]{article}
\usepackage{polski}
\usepackage[utf8]{inputenc}
\usepackage{amsmath}
\usepackage{graphicx}
\usepackage{geometry}
\usepackage{float}

\geometry{legalpaper, margin=0.6in}

%opening
\title{Projekt MORO}
\author{Jakub Postępski}
\date{3 stycznia 2019}

\begin{document}

\maketitle

\section{Robot}

Robot zaprezentowany na wykładzie (rys. ) to ABB IRB 6620.

\section{Parametry DH}

\begin{table}[H]
	\begin{tabular}{|| c | c c c c ||}
		\hline
		L. p. & $a_{i-1}$ & $\alpha_{i-1}$ & $d_i$ & $\theta_i$ \\ 
		\hline\hline
		1 & $ 0 $ & $0$ & 0 & $\theta_1$ \\
		\hline
		2 & $a_1$ & $-\pi/2$ & 0 & $\theta_1$ \\
		\hline
		3 & $a_2$ & $0$ & 0 & $\theta_1$ \\
		\hline
		4 & $a_3$ & $-\pi/2$ & $a_4$ & $\theta_1$ \\
		\hline
		5 & $0$ & $\pi/2$ & 0 & $\theta_1$ \\
		\hline
		6 & $0$ & $-\pi/2$ & 0 & $\theta_1$ \\
		\hline
		
		
		
	\end{tabular}
	\caption{Parametry DH}
	\label{tab1}
\end{table}

\section{Kinematyka prosta}

\section{Kinematyka odwrotna}

\end{document}