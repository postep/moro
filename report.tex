\documentclass[]{article}
\usepackage{polski}
\usepackage[utf8]{inputenc}
\usepackage{amsmath}
\usepackage{mathtools,leftidx}% http://ctan.org/pkg/{mathtools,leftidx}
\usepackage{graphicx}
\usepackage{geometry}
\usepackage{float}

\geometry{legalpaper, margin=0.6in}

%opening
\title{Projekt MORO}
\author{Jakub Postępski}
\date{3 stycznia 2019}

\begin{document}

\maketitle


\section{Parametry DH}

\begin{table}[H]
	\begin{tabular}{|| c | c c c c ||}
		\hline
		L. p. & $a_{i-1}$ & $\alpha_{i-1}$ & $d_i$ & $\theta_i$ \\ 
		\hline\hline
		1 & $ 0 $ & $0$ & 0 & $\theta_1$ \\
		\hline
		2 & $a_1$ & $\pi/2$ & 0 & $\theta_2$ \\
		\hline
		3 & $a_2$ & $0$ & 0 & $\theta_3$ \\
		\hline
		4 & $a_3$ & $\pi/2$ & $a_4$ & $\theta_4$ \\
		\hline
		5 & $0$ & $\pi/2$ & 0 & $\theta_5$ \\
		\hline
		6 & $0$ & $\pi/2$ & 0 & $\theta_6$ \\
		\hline
		
		
		
	\end{tabular}
	\caption{Parametry DH}
	\label{tab1}
\end{table}

\section{Kinematyka prosta}
Przejścia pomiędzy członami:
\[\prescript{0}{1}{T}=\begin{bmatrix}
\begin{array}{cccc}
c_1 & -s_1 & 0 & 0 \\
s_1 & c_1 & 0 & 0 \\
0 & 0 & 0 & 0 \\
0 & 0 & 0 & 1 \\
\end{array}
\end{bmatrix}
\]
\[\prescript{1}{2}{T}=\begin{bmatrix}
\begin{array}{cccc}
c_2 & -s_2 & 0 & a_1 \\
0 & 0 & -1 & 0 \\
s_2 & c_2 & 0 & 0 \\
0 & 0 & 0 & 1 \\
\end{array}
\end{bmatrix}
\]
\[\prescript{2}{3}{T}=\begin{bmatrix}
\begin{array}{cccc}
c_3 & -s_3 & 0 & a_2 \\
s_3 & c_3 & 0 & 0 \\
0 & 0 & 1 & 0 \\
0 & 0 & 0 & 1 \\
\end{array}
\end{bmatrix}
\]
\[\prescript{3}{4}{T}=\begin{bmatrix}
\begin{array}{cccc}
c_4 & -s_4 & 0 & a_3 \\
0 & 0 & -1 & -d_4 \\
s_4 & c_4 & 0 & 0 \\
0 & 0 & 0 & 1 \\
\end{array}
\end{bmatrix}
\]

\[\prescript{4}{5}{T}=\begin{bmatrix}
\begin{array}{cccc}
c_5 & -s_5 & 0 & 0 \\
0 & 0 & -1 & 0 \\
s_5 & c_5 & 0 & 0 \\
0 & 0 & 0 & 1 \\
\end{array}
\end{bmatrix}
\]

\[\prescript{5}{6}{T}=\begin{bmatrix}
\begin{array}{cccc}
c_6 & -s_6 & 0 & 0 \\
0 & 0 & -1 & 0 \\
s_6 & c_6 & 0 & 0 \\
0 & 0 & 0 & 1 \\
\end{array}
\end{bmatrix}
\]

Po wymnożeniu:
\[\prescript{0}{6}{T}=\begin{bmatrix}
\begin{array}{cccc}
r_{11} & r_{12} & r_{13} & p_x \\
r_{21} & r_{22} & r_{23} & p_y \\
r_{31} & r_{32} & r_{33} & p_z \\
0 & 0 & 0 & 1 \\
\end{array}
\end{bmatrix}
\]
gdzie:
\[ r_{11} = c_1(c_{23}(c_4c_5c_6 + s_4s_6) + s_{23}s_5c_6) - s_1(c_4s_6 -s_4c_5c_6)\]
\[ r_{12} = c_1(c_{23}(s_4c_6 - c_4c_5c_6) - s_{23}s_5s_6) - s_1(s_4c_5s_6 + c_4s_6)\]
\[ r_{13} = c_1(c_{23}c_4s_5) + s_1s_4s_5\]
\[ r_{21} = s_1(c_{23}(c_4c_5c_6 + s_4s_6) + s_{23}s_5c_6) + c_1(c_4s_6 -s_4c_5c_6 )\]
\[ r_{22} = s_1(c_{23}(s_4c_6 - c_4c_5c_6) - s_{23}s_5s_6) + c_1(s_4c_5s_6 + c_4c_6)\]
\[ r_{23} = s_1(c_{23}c_4s_5 - s_{23}c_5) - c_1s_4s_5\]
\[ r_{31} = s_{23}(c_4c_5c_6 + s_4s_6) - c_{23}s_5c_6\]
\[ r_{32} = s_{23}(s_4c_6 - c_4c_5c_6) + c_{23}s_5s_6\]
\[ r_{33} = s_{23}c_4s_5 + c_{23}c_5\] 
\[ p_x = c_1(c_{23}a_3 + s_{23}d_4 + c_2a_2 + a_1))\] 
\[ p_y = s_1(c_{23}a_3 + s_{23}d_4 + c_2a_2 + a_1))\]
\[ p_z = s_{23}a_3 - c_{23}d_4 + s_2a_2 \]

\section{Kinematyka odwrotna}
Korzystamy z zależności
\[ \prescript{}{}{T}_d = \prescript{0}{6}{T} = \prescript{0}{1}{T} \cdot \prescript{1}{6}{T} \]
więc:
\[\prescript{0}{1}{T}^{-1} \cdot \prescript{}{}{T}_d = \prescript{1}{6}{T} \]
\[ \prescript{0}{1}{T}^{-1} \cdot \prescript{}{}{T}_d =  \begin{bmatrix}
\begin{array}{cccc}
c_1 & s_1 & 0 & 0 \\
-s_1 & c_1 & 0 & 0 \\
0 & 0 & 0 & 0 \\
0 & 0 & 0 & 1 \\
\end{array}
\end{bmatrix} \cdot  \begin{bmatrix}
\begin{array}{cccc}
r_{11} & r_{12} & r_{13} & p_x \\
r_{21} & r_{22} & r_{23} & p_y \\
r_{31} & r_{32} & r_{33} & p_z \\
0 & 0 & 0 & 1 \\
\end{array}
\end{bmatrix}  = [] \]

TUTAJ UDOSTEPNI MJB ALE TRZEBA POZMIENIAC MINUSY
\subsection{$\theta_1$}
Bierzemy wyrażenie w 2 wierszu i 4 kolumnie z lewej strony i przyrównujemy do zera z prawej:
\[ -p_xs_1 + p_yc_1 = 0 \]
\[ p_yc_1 = p_xs_1 \]
\[ \frac{s_1}{c_1} = \frac{p_y}{p_x} \]
\[ \theta_1 = \arctan(\frac{p_y}{p_x})\]
\subsection{$\theta_2$}
Bierzemy z wiersza 1 i kolumny 4 oraz wiersza 3 i kolumny 4:
\[ \left\{\begin{array}{c}
c_{23}a_3 + s_{23}d_4 + c_2a_2 + a_1 = p_x c_1 + p_y s_1 \\
s_{23}a_3 - c_{23}d_4+s_2a_2 = p_x 
\end{array} \right. \]

wprowadzamy stałe:
\[ E = p_xc_1 + p_ys_1 -a_1\]
\[F = p_x\]

i dostajemy układ:
\[ \left\{\begin{array}{c}
c_{23}a_3 + s_{23}d_4 = E - c_2a_2\\
s_{23}a_3 - c_{23}d_4 = F - s_2a_2 
\end{array} \right. \]

stąd:
\[ \left\{\begin{array}{c}
(c_{23}a_3)^2 + 2c_{23}a_3s_{23}d_4 + (s_{23}d_4)^2 = E^2 -2Ec_2a_2 (c_2a_2)^2 \\
(s_{23}a_3)^2 -2s_{23}a_3c_{23}d_4 + (c_{23}d_4)^2 = F^2 - 2*Fs_2a_2 + (s_2a_2)^2 
\end{array} \right. \]

po dodaniu stronami i zastosowaniu jedynki trygonometrycznej:
\[ a_3^2 + d_4^2 = E^2 + F^2 -2Ec_2a_2 - 2Fs_2a_2 + a_2^2 \]

po uporządkowaniu możemy podstawić:
\[ K = E^2 + F^2 + a_2^2 - a_3^2 - d_4^2 \]
\[2Ec_2a_2 + 2Fs_2a_2 = K \]

i możemy zastosować wsp. biegunowe poprzez:
\[ \left\{\begin{array}{c}
2Ea_2 = Rs_\phi \\
2Fa_2 = Rc_\phi
\end{array} \right. \]

\[ R = \sqrt{4E^2a_2^2 + 4F^2a_2^2} \]
\[ K = Rs_\phi c_2 + Rc_\phi s_2 \]

wyznaczamy ze wzorów na sumy trygonometryczne:
\[ K = Rs_{(\phi + \theta_2)} \]
a z jedynki trygonometrycznej:
\[ c_{(\phi + \theta_2)} = \pm\sqrt{1-(\frac{K}{R})^2} \]
więc:
\[ \tan{\phi + \theta_2}  = \frac{K/R}{\pm \sqrt{1-(\frac{K}{R})^2}}\]
\[ \phi + \theta_2  = \arctan{\frac{K/R}{\pm \sqrt{1-(\frac{K}{R})^2}}}\]
\[ \phi  = \arctan{\frac{K/R}{\pm \sqrt{1-(\frac{K}{R})^2}}} - arctan{\frac{2Ea_2}{2Fa_2}}\]
\subsection{$\theta_3$}
\subsection{$\theta_4$}
\subsection{$\theta_5$}
\subsection{$\theta_6$}

\end{document}